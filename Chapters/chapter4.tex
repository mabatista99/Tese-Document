%!TEX root = ../template.tex
%%%%%%%%%%%%%%%%%%%%%%%%%%%%%%%%%%%%%%%%%%%%%%%%%%%%%%%%%%%%%%%%%%%%
%% chapter4.tex
%% NOVA thesis document file
%%
%% Conclusions chapter
%%%%%%%%%%%%%%%%%%%%%%%%%%%%%%%%%%%%%%%%%%%%%%%%%%%%%%%%%%%%%%%%%%%%

\typeout{NT FILE chapter4.tex}

\chapter{Conclusions}
\label{cha:conclusions}

The increasing adoption of \gls{IoT} systems for monitoring and surveillance
comes with new challenges in scalability, processing large volume of data in
real time, security and anomaly detection efficiency.

The purpose of this document was to research and analyse the current state-of-the-art
of \gls{IoT} platforms for monitoring and surveillance. The goal is to develop
a robust, scalable and generic platform capable of addressing the challenges
inherent to this systems. In order to do that, ways to detect anomalies were
examined, starting with the traditional and simpler methods and moving to the
more complex approaches using \gls{AI} and \gls{ML}.

After that the system
architectures were analysed in detail. Event-driven architectures using
edge computing should be appropriate to fill the project requirements. The next
step was finding the best tools to develop the backend platform, concluding that
each framework has its own pros and cons, but all of them can fill the needs.

In the next phase, the storage problem was address, and several databases were examined.
With some research, the conclusion was that an hybrid approach might be the best
choice, with different types of databases being used for different types of data.

Altough the focus is on the backend platform, in order to test the platform,
a frontend page will be required and some frontend frameworks were then briefly
analysed. Just like the backend frameworks, all the frontend frameworks are
suitable for the requirements.

Lastly, there was the need to explore how data can be transform and analysed,
analysing data processing pipelines and diving into concepts like \gls{ETL},
and \gls{BI}.

To conclude, there are several approaches to the challenges of \gls{IoT} monitoring
and surveillance platforms, each with its own set of benefits and drawbacks.
More research can be needed in the future to complement the work already done.
