%!TEX root = ../template.tex
%%%%%%%%%%%%%%%%%%%%%%%%%%%%%%%%%%%%%%%%%%%%%%%%%%%%%%%%%%%%%%%%%%%
%% chapter1.tex
%% NOVA thesis document file
%%
%% Chapter with introduction
%%%%%%%%%%%%%%%%%%%%%%%%%%%%%%%%%%%%%%%%%%%%%%%%%%%%%%%%%%%%%%%%%%%

\typeout{NT FILE chapter1.tex}%

\chapter{Introduction}
\label{cha:introduction}

\prependtographicspath{{Chapters/Figures/Covers/}}
\section{Context and Motivation}
In the last years, the digitalization of society has led to an increased
demand on \gls{IoT} systems. Whether the goal is equipment monitoring, traffic
management, security, or industrial automation, \gls{IoT} systems are a more
scalable and efficient alternative than the traditional methods.

\gls{IoT} systems are networks of \gls{IoT} devices connected to each other
that can collect, process, and transmit data in real-time.

Real-time monitoring requires processing efficiently a continuous flow of data from many
devices, resulting in a large volume of data. This data must be processed
quickly and accurately so the system can detect anomalies like equipment failures
efficiently.

IoT platforms face unique challenges like handling large volumes of
unstructured data, managing connectivity problems, and ensuring low-latency
processing. As the system grows, the number of \gls{IoT} devices connected to
the platform increases, introducing challenges related to scalability, data
security, and reliability.

To address these challenges, cloud architectures, allied to edge computing, provide highly scalable solutions
that allow efficient data collection, seamless device integration, and powerful
computational resources.

Additionally, \gls{IoT} systems have been relying more and more on \gls{AI} and \gls{ML} for
anomaly detection and predictive analytics, yet deploying complex \gls{AI}
models to resource-constrained \gls{IoT} devices remains a complex process.

\section{Problem}

As digitalization advances, monitoring and surveillance requirements are
becoming more and more complex, requiring scalable, highly customizable and
low-cost solutions.

Traditional monitoring platforms are often rigid and limited in their ability
to scale and adapt to new requirements, like new devices with different data.
The high processing capacity required for managing many \gls{IoT} devices,
ensuring real-time processing, and integrating predictive \gls{AI}/\gls{ML}
analytics emphasizes the need for more advanced architectures.

The increased digitalization in sectors like manufacturing, smart cities,
and agriculture requires an open, generic, and scalable platform that takes
advantage of the cloud and edge computing scaling benefits, is able to
adapt to diverse anomaly detection scenarios, and supports integration with
\gls{AI}/\gls{ML}.
\section{Proposed Solution}

The goal of this project is to develop and document a robust and scalable
\gls{IoT} backend platform that interacts with \gls{IoT} devices for monitoring and
surveillance. These devices are programmed to detect a variety of different
anomalies, like equipment failures, building alarms, and traffic violations.

The platform will be done recurring to cloud and edge computing, ensuring
scalability and flexibility. The data received from the devices will be stored
in the adequate database system.

Additionally, the platform will be extensible, allowing the integration of new
modules like \gls{ETL} for \gls{IoT}, big data analytics, business intelligence,
and \gls{AI}/\gls{ML}.

Other modules, for example, monitoring and alarm modules, can also be added, as
well as rule engines.

This project will also include the development of a simple IoT device and a
frontend web page for testing purposes.

\section{Dissertation Outline}
The structure of the document will be as follows:

\begin{description}
	\item[Chapter 1 ] This chapter briefly introduces the problem, as well as the
	      solution proposed in this document.
	\item[Chapter 2 ] The second chapter presents the state-of-the-art of event
	      management platforms for IoT, describing anomaly detection methods, system architectures, frameworks,
	      communication, data pipelines, and \gls{AI}/\gls{ML} integration.
	\item[Chapter 3 ] In the third chapter, a description of the tasks and
	      schedule for the following period is presented.
	\item[Chapter 4 ] Lastly, the conclusions on the state-of-the-art are presented.
\end{description}
